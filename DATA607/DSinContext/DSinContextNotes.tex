% Created 2019-09-24 Tue 12:11
% Intended LaTeX compiler: pdflatex
\documentclass[11pt]{article}
\usepackage[utf8]{inputenc}
\usepackage[T1]{fontenc}
\usepackage{graphicx}
\usepackage{grffile}
\usepackage{longtable}
\usepackage{wrapfig}
\usepackage{rotating}
\usepackage[normalem]{ulem}
\usepackage{amsmath}
\usepackage{textcomp}
\usepackage{amssymb}
\usepackage{capt-of}
\usepackage{hyperref}
\author{Jai Jeffryes}
\date{\today}
\title{}
\hypersetup{
 pdfauthor={Jai Jeffryes},
 pdftitle={},
 pdfkeywords={},
 pdfsubject={},
 pdfcreator={Emacs 25.2.1 (Org mode 9.0.10)}, 
 pdflang={English}}
\begin{document}

\tableofcontents

\section{Waving from Sulzer Lab}
\label{sec:org9661739}
\subsection{Dr. Sulzer.}
\label{sec:orgef1a641}
\subsection{Research DA system. PD, schizophrenia, drugs of abuse.}
\label{sec:orgfff9e56}
\subsection{Support Dr. Subramanji. Amphetemine and DA.}
\label{sec:orge5beeee}
\section{Vesicle attribution}
\label{sec:orge6e0431}
By Nrets - first upload on en.wikipedia.org, uploaded to Wikimedia Commons as Image:SynapseIllustration2.png, SVG version by User:Surachit. For further information on the image's contents, see Julien, R. M. (2005). The neuron, synaptic transmission, and neurotransmitters. In R. M. Julien, A primer of drug action: A comprehensive guide to the actions, uses, and side effects of psychoactive drugs (pp. 60-88). New York, NY, USA: Worth Publishers., CC BY-SA 3.0, \url{https://commons.wikimedia.org/w/index.php?curid=4001388}
\section{Neuron attribution}
\label{sec:org1b170ea}
Public Domain, \url{https://commons.wikimedia.org/w/index.php?curid=254226}
\section{Script}
\label{sec:org7335087}
1/9
I'm Jai Jeffryes and I'm an analyst and programmer at the Sulzer
Neuroscience Laboratory at Columbia University.

2/9 I'm waving to you from the lab (while getting photobombed). On the
left of the screen is Dr. David Sulzer, the principal investigator of
the lab and my boss. The lab researches the dopamine system, which
figures prominently in Parkinson's Disease, schizophrenia, and drugs
of abuse. On the right of the screen is Dr. Mahalakshmi
Somayaji. Since I joined the lab in March, my main responsibility has
been supporting Maha's research into the effects of amphetamine on
the neurotransmitter, dopamine.

3/9
Signal propagates in a neuron from dendrites, seen on the left, to the
axon terminals, detailed on the right. The bubbles in the axon
terminal represent synaptic vesicles, tiny sacks containing molecules
of dopamine. They transmit dopamine by fusing with the plasma membrane
and discharging their cargo into the synaptic cleft. A few such
molecules are depicted here. Some of those will bind with
neurotransmitter receptors in the dendritic spines of the adjacent
neuron. The rest will be pumped back up into the axon terminal. Keep
in mind these dual processes of release and reuptake.

4/9 We can model the kinetics of dopamine in a one-dimensional random
walk of its diffusion. The release locations are modeled here from
left to right. Here I release 2.75 micromolar of dopamine. Time slices
proceed downwards on the matrix. In the next time slice, half of the
molecules will diffuse to the left neighbor and half to the
right. Then reuptake reduces the concentration. Now, the next site
will diffuse to the left and right in the time slice after that. I cut
out a lot of the matrix. It propogates like this to a column
representing a measuring electrode, where you see the concentration of
dopamine over time.

5/9
The reduction in dopamine from reuptake is given by the
Michaelis-Menten equation. Note the quantity, km, which expresses the
affinity between dopamine and its target transporter. If km is
lowered, the reuptake increases.

6/9 Thus, I can plot a simulation of dopamine concentration. We see
release followed by reuptake. If I reduce km from 2 to .8, the decay
of the concentration of dopamine is more rapid in the model.

7/9 This is a plot of the data I receive from Maha. She stimulates a
mouse brain every two minutes and it evokes dopamine release. Animal
research can be a sensitive subject. I can tell you that it is highly
regulated, the animals are anesthetized and they feel nothing.

Note the change in height of the peaks. Here, in the fourth
stimulation, is where Maha administers amphetamine. Amphetamine both
raises dopamine release and reduces its reuptake, but how much of
each?  You can't tell from this plot, for the concentration is the net
of the two.

Maha believes she has learned something novel about the mechanism of
amphetamine's influence on dopamine release. To demonstrate it, she
needs to isolate and quantify dopamine's release and reuptake.

8/9
We do this by superimposing the model on the data from a single
stimulus. The model is the green line, Maha's data is red. My first
estimate isn't very good. I've underestimated the release, so the peak
height is too low, and I've underestimated the reuptake so the curve
is spread too widely. I adjust the two until I find a good fit. The
correlation is quantified by the statistic r-squared.

9/9
I've inferred measures of release and reuptake based on the model, and
hopefully these estimates are sufficiently reliable for Maha to draw
conclusions about the mechanism of dopamine kinetics. Her hypothesis
involves a particular protein playing a role in vesicular fusion. Her
control for the experiment is so-called knock out mice, animals
genetically engineered to lack the protein of interest.

Whatever Maha learns, she'll publish, and I will join the byline as
one of the co-authors. My contribution is an R package hosted on
GitHub and my analysis.

That concludes my presentation, Dr. Catlin.
\end{document}